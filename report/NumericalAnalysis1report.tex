\documentclass{jarticle}
\usepackage{amsmath,amssymb}
\usepackage[dvipdfmx]{graphicx}
\usepackage{cases}
%%%%%%%%%%%%%%%%%%%%%%%%%%%%%%%%%%%%%%%%%%%%%%%%%%%%%%%%%%%%
% 使いたいパッケージがあればこの辺りに書いてください
%\usepackage{empheq}
%%%%%%%%%%%%%%%%%%%%%%%%%%%%%%%%%%%%%%%%%%%%%%%%%%%%%%%%%%%%
% マクロが使いたいときはこの辺りに書いてください
\newcommand{\prob}[1]{\fbox{\Large \,#1\,}}
\newcommand{\ans}[0]{\fbox{解答}\\}
%%%%%%%%%%%%%%%%%%%%%%%%%%%%%%%%%%%%%%%%%%%%%%%%%%%%%%%%%%%%
% レポート提出の際は title, author, date を適当に変更してください
\title{数値解析特論I 期末レポート}
\author{担当教員: 木下武彦}
\date{確定版}
\begin{document}
\maketitle
%%%%%%%%%%%%%%%%%%%%%%%%%%%%%%%%%%%%%%%%%%%%%%%%%%%%%%%%%%%%
\prob{1} 第2回 -- 第4回の講義資料にあるグループワーク課題のどれか1つを解いてください.

\textbf{[補足]}
\begin{itemize}
\item 物理モデル(単位を持つ偏微分方程式)とそれを無次元化した数学モデル(無次元量のみから成る偏微分方程式)の両方を導出してください.
\end{itemize}

\ans{}
第2回の課題1を解いた.スライド中の仮定を用いて,微小部分にかかる力は,
\begin{eqnarray}
  F(X + \Delta X, T) \sin\theta(X + \Delta X, T) - F(X, T) \sin\theta(X, T)
  \label{braEq}
\end{eqnarray}
この微小部分における運動方程式は,$\rho$が$X$に依存するため,以下のように求まる.
\begin{eqnarray}
  \Delta X( \rho(X+\Delta X) - \rho(X) ) \frac{ \partial^2 U }{ \partial T^2 }(X,T) \nonumber \\
  = F(X + \Delta X, T) \sin\theta(X + \Delta X, T) - F(X, T) \sin\theta(X, T)
  \label{moEq}
\end{eqnarray}
式\ref{moEq}を変形し,$\Delta X \rightarrow 0$の極限を取ると,
\begin{eqnarray*}
  \frac{ \partial^2 U }{ \partial T^2 }(X,T) &=&
  \frac{1}{\Delta X} \left( \frac{ F(X + \Delta X, T) \sin\theta(X + \Delta X, T) - F(X, T) \sin\theta(X, T) }{ \rho(X+\Delta X) - \rho(X) } \right) \\
  &=& \frac{\partial}{\partial X} \left( \frac{F}{\rho} \sin\theta(X, T) \right)
\end{eqnarray*}
次に,右辺の分母・分子に$\frac{1}{\cos\theta(X, T)}$をかけて,
$\cos^2\theta = \frac{1}{1 + \tan^2\theta}$より,
\begin{eqnarray*}
  &=& \frac{\partial}{\partial X} \left( \frac{ \frac{F}{\rho} \frac{\partial U}{\partial X}(X,T) }{\frac{1}{\cos\theta(X, T)}} \right) \\
  &=& \frac{\partial}{\partial X} \left( \frac{ \frac{F}{\rho} \frac{\partial U}{\partial X}(X,T) }{\sqrt{1 + \left( \frac{\partial U}{\partial X}(X,T) \right)^2 }} \right)
\end{eqnarray*}
よって,以下の非線形偏微分方程式が求まる.
\begin{eqnarray}
  \frac{ \partial^2 U }{ \partial T^2 }(X,T) = \frac{\partial}{\partial X} \left( \frac{ \frac{F}{\rho} \frac{\partial U}{\partial X}(X,T) }{\sqrt{1 + \left( \frac{\partial U}{\partial X}(X,T) \right)^2 }} \right)
  \label{parEq}
\end{eqnarray}
次に,式\ref{parEq}を無次元化する.スライドと同様に,$X$の取りえる範囲を$\left( X_L, X_R \right)$,
$T$の取りえる範囲を$\left( 0, \infty \right)$とし,定数$X_{0} > 0$を取り,$X_0 := X_R - X_L$とする.
また,$F(X, T)と\rho(X)$を無次元化するために,$X$同様,取りえる範囲をそれぞれ,$(F_L, F_R),(\rho_L, \rho_R)$と定め,
定数$F_0, \rho_0 > 0$を取り,$F_0 := F_R - F_L, \rho_0 := \rho_R - \rho_L$とする.
以下に,無次元化された変数を列挙する.
\begin{eqnarray*}
  u &:=& \frac{U}{X_0}(X, T) \qquad [m]\cdot[1/m] = [1] \\
  x &:=& \frac{X - X_L}{X_0} \qquad [m]\cdot[1/m]  = [1] \\
  f &:=& \frac{F}{F_0}(X, T) \qquad [N]\cdot[1/N]  = [1] \\
  r &:=& \frac{\rho}{\rho_0}(X) \qquad [kg/m]\cdot[m/kg] = [1] \\
  t &:=& \frac{1}{X_0}\sqrt{\left( \frac{F_0}{\rho_0} \right)} T \qquad [1/m]\cdot[m/s]\cdot[s] = 1 
\end{eqnarray*}
このとき,$u(x,t)$の満たす微分方程式は以下のようになる.
\begin{eqnarray*}
  \frac{\partial u}{\partial t}(x,t) &=& \frac{1}{X_0} \frac{\partial U}{\partial T}(X,T) \frac{dT}{dt}(t) = \sqrt{\frac{\rho_0}{F_0}} \frac{\partial U}{\partial T}(X,T) \\
  \frac{\partial^2 u}{\partial t^2}(x,t) &=& X_0 \frac{\rho_0}{F_0} \frac{\partial^2 U}{\partial T^2}(X,T) \\
  \frac{\partial u}{\partial x}(x,t) &=& \frac{\partial U}{\partial X}(X,T) \\
  \frac{\partial^2 u}{\partial x^2}(x,t) &=& X_0 \frac{\partial^2 U}{\partial X^2}(X,T)
\end{eqnarray*}
これらの関係を用いて,式\ref{parEq}を変形すると,
\begin{eqnarray*}
  \frac{ \partial^2 U }{ \partial T^2 }(X,T) &=& \frac{\partial}{\partial X} \left( \frac{ \frac{F}{\rho} \frac{\partial U}{\partial X}(X,T) }{\sqrt{1 + \left( \frac{\partial U}{\partial X}(X,T) \right)^2 }} \right) \\
  \frac{1}{X_0} \frac{F_0}{\rho_0} \frac{\partial^2 u}{\partial t^2}(x,t) &=& \frac{\partial}{\partial X} \left( \frac{ \frac{F}{\rho} \frac{\partial U}{\partial X}(X,T) }{\sqrt{1 + \left( \frac{\partial U}{\partial X}(X,T) \right)^2 }} \right) \\
  \frac{1}{X_0} \frac{\partial^2 u}{\partial t^2}(x,t) &=& \frac{\partial}{\partial X} \left( \frac{ \frac{F}{F_0}\frac{\rho_0}{\rho}\frac{\partial U}{\partial X}(X,T) }{\sqrt{1 + \left( \frac{\partial U}{\partial X}(X,T) \right)^2 }} \right) \\
  &=& \frac{\partial}{\partial X} \left( \frac{ \frac{\partial U}{\partial X}(X,T) }{\sqrt{1 + \left( \frac{\partial U}{\partial X}(X,T) \right)^2 }} \right) \frac{f}{r} \\
  &=& \frac{ \frac{\partial^2 U}{\partial X^2} }{ \frac{\partial}{\partial X} \sqrt{1 + \left( \frac{\partial U}{\partial X}(X,T) \right)^2 } } \frac{f}{r} \\
  &=& \frac{1}{X_0} \frac{f}{r} \sqrt{1 + \left( \frac{\partial u}{\partial x}(x,t) \right)^2}
\end{eqnarray*} 
よって,無次元化された偏微分方程式は以下のようになる.
\begin{eqnarray}
  \frac{\partial^2 u}{\partial t^2}(x,t) = \frac{f}{r} \sqrt{1 + \left( \frac{\partial u}{\partial x}(x,t) \right)^2}
  \label{noDim}
\end{eqnarray}
\\
%%%%%%%%%%%%%%%%%%%%%%%%%%%%%%%%%%%%%%%%%%%%%%%%%%%%%%%%%%%%
\prob{2} $\Omega := (0,1) \times (0,1)$ とします.
\[ f(x,y) := \left(x-\frac{1}{2}\right)e^{-5\sqrt{\left(x-\frac{1}{2}\right)^2+\left(y-\frac{1}{2}\right)^2}}, \quad \forall (x,y) \in \Omega \]
のグラフを描画してください.

\textbf{[補足]}
\begin{itemize}
\item グラフを描画する方法は NumPy + matplotlib, SymPy, $f$ の区分1次補間など,何を使っても構いません.
\item 区分1次補間のグラフを描画する場合は,R20 メッシュを使ってください.
\item 解答はグラフを描画した画像を貼り付けてください.
\end{itemize}
%%%%%%%%%%%%%%%%%%%%%%%%%%%%%%%%%%%%%%%%%%%%%%%%%%%%%%%%%%%%
\prob{3} $\phi_1(x) := \sin(\pi x)$, $\phi_2(x) := \sin(2 \pi x)$ とおきます.
$\{\phi_i\}_{i=1}^2$ を基底とする関数空間を
\[ V_h := \left\{\alpha_1\phi_1+\alpha_2\phi_2 \in H_0^1(0,1) \,\middle|\, \forall \alpha_1,\alpha_2 \in \mathbb{R}\right\} \]
とします.
このとき,任意に与えられた $V_h \ni f_h := c_1\phi_1 + c_2\phi_2$, $(c_1,c_2 \in \mathbb{R})$ に対して,次の変分問題:
\begin{align}
\int_0^1\frac{du_h}{dx}(x)\frac{dv_h}{dx}(x)\,dx = \int_0^1f_h(x)v_h(x)\,dx, \quad \forall v_h \in V_h \label{prob3}
\end{align}
の解 $u_h \in V_h$ を求めてください.

\textbf{[補足]}
\begin{itemize}
\item $u_h = a_1\phi_1 + a_2\phi_2$, $(a_1,a_2 \in \mathbb{R})$, $v_h = b_1\phi_1 + b_2\phi_2$, $(b_1,b_2 \in \mathbb{R})$ として,これらを \eqref{prob3} に代入します.
\item 積分は手で計算しても構いませんし,SymPy を使っても構いません.
\item \eqref{prob3} は任意の $v_h \in V_h$ に対して等号が成り立つ $\Longleftrightarrow$ 任意の実数 $b_1,b_2 \in \mathbb{R}$ について積分を計算した等号が成り立つ $\Longleftrightarrow$ $b_1,b_2$ に関する恒等式となることから,\eqref{prob3} と同値な連立一次方程式がえられます.
\item 連立一次方程式を解けば $u_h$ が求まります.具体的に,$a_1,a_2$ は $c_1,c_2$ を用いて書き下すことができます.
\end{itemize}
%%%%%%%%%%%%%%%%%%%%%%%%%%%%%%%%%%%%%%%%%%%%%%%%%%%%%%%%%%%%
\prob{4} $\Omega := (0,1) \times (0,1)$ の正方形領域とします.
\[ g_D(x,y) := \begin{cases}
y(1-y), &\forall (1,y) \in \partial\Omega \\
0, &\textrm{otherwise}
\end{cases} \]
とおきます.
このとき,次の Poisson 方程式:
\[ \begin{cases}
-\triangle u = 0 &\textrm{in}~\Omega \\
u = g_D &\textrm{on}~\partial\Omega
\end{cases} \]
の有限要素解を求めてください.

\textbf{[補足]}
\begin{itemize}
\item $\Omega$ のメッシュは講義資料で配布した R20 メッシュを使ってください.
\item 解答は有限要素解のグラフを描画した画像を貼り付けてください.
\end{itemize}
%%%%%%%%%%%%%%%%%%%%%%%%%%%%%%%%%%%%%%%%%%%%%%%%%%%%%%%%%%%%
\end{document}









